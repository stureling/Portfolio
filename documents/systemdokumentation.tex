\documentclass{TDP003mall}
\usepackage{titlesec}
\definecolor{terminalgreen}{HTML}{8AE234}

\newcommand{\version}{Version 0.1}
\author{Eric Jönsson, \url{erijo137@student.liu.se}\\
  Ida Bergquist, \url{idabe112@student.liu.se}}
\title{Systemdokumentation}
\date{2018-10-16}
\rhead{Ida Bergquist\\
Eric Jönsson}



\begin{document}
\projectpage
\tableofcontents
\newpage
\section{Revisionshistorik}
\begin{table}[!h]
\begin{tabularx}{\linewidth}{|l|X|l|}
\hline
\textbf{Ver.} & \textbf{Revisionsbeskrivning} & \textbf{Datum} \\\hline
0.1 & Skapade systemdokumentationen & 181016 \\\hline
\end{tabularx}
\end{table}

\section{Overview}
\subsection{S}
\section{Design}
\subsection{Data API}

\subsubsection{Functions}
\paragraph{load}
\paragraph{save}
\paragraph{load\_users}
\paragraph{get\_project\_count}
\paragraph{get\_project}
\paragraph{search}
\paragraph{get\_techniques}
\paragraph{get\_searchfields}
\paragraph{get\_technique\_stats}

\subsubsection{JSON}
\paragraph{Data}
\paragraph{Users}

\subsection{User interface}

\subsubsection{main.py}

\subsubsection{forms.py}

\subsubsection{login.py}

\section{Errors and errorlogs}
Logging av felmeddelanden sker via terminalen. Ifall det blir fel i jinja kommer felmeddelandet också till webbläsaren.


\end{document}
