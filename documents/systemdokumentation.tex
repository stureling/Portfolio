\documentclass{TDP003mall}
\usepackage{titlesec}
\usepackage{graphicx}
\definecolor{terminalgreen}{HTML}{8AE234}

\newcommand{\version}{Version 0.1}
\author{Eric Jönsson, \url{erijo137@student.liu.se}\\
  Ida Bergquist, \url{idabe112@student.liu.se}}
\title{Systemdocumentation}
\date{2018-10-16}
\rhead{Ida Bergquist\\
Eric Jönsson}



\begin{document}
\projectpage
\tableofcontents
\newpage
\section{Revisions}
\begin{table}[!h]
\begin{tabularx}{\linewidth}{|l|X|l|}
\hline
\textbf{Ver.} & \textbf{Description of revisions} & \textbf{Datum} \\\hline
0.1 & First version created & 181016 \\\hline
\end{tabularx}
\end{table}

\section{Overview}
\begin{figure}[h!]
    \centering
    \includegraphics[width=10cm]{sevenskdiagram.png}
    \caption{Simplified view of the portfolio}
    \label{sekvensdiagram}
\end{figure}
In broad terms, the portfolio functions by the user interface taking request from the user and then sends a query to the data API for data needed to create the reqested html-page.
The user interface sends specific requests to the data API depending on what data is needed to build the html-page.
\section{Design}
\subsection{Data API}
All functions are defined in a seperate document called module\_documentation.pdf.
\subsection{User interface}
\begin{figure}[h!]
    \centering
    \includegraphics[width=14cm]{sekvensdiagram2-3.png}
    \caption{The inner workings of the user interface}
    \label{sekvensdiagram2}
\end{figure}
All functions are defined in a seperate document called module documentation.

\newpage
\section{Errors and errorlogs}
\subsection{Logs}
When the project is running logs are printed out directly to the terminal.
No file logging is performed.

\subsection{Tests}
The Data API is built to handle the unit tests present at: \href{https://gitlab.ida.liu.se/filst04/tdp003-2018-database-tests}{Database Tests}. These
tests can be cloned into a suitable folder on a local computer and ran to
verify that the Data API works as intended. Please note that in order to run
the tests, the data.py file from the project has to be present in the
same folder as the data\_test.py file from the database test repo.
To run the tests simply execute data\_test.py within your terminal.

No specific tests are executed on the front end itself, however one might
concievably want to write some on one's own.

\end{document}
