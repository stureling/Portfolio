\documentclass{TDP003mall}
\definecolor{terminalgreen}{HTML}{8AE234}

\newcommand{\version}{Version 0.1}
\author{Eric Jönsson, \url{erijo137@student.liu.se}\\
  Ida Bergquist, \url{idabe112@student.liu.se}}
\title{Projektplan}
\date{2018-09-25}
\rhead{Ida Bergquist\\
Eric Jönsson}



\begin{document}
\projectpage
\tableofcontents
\newpage
\section{Revisionshistorik}
\begin{table}[!h]
\begin{tabularx}{\linewidth}{|l|X|l|}
\hline
\textbf{Ver.} & \textbf{Revisionsbeskrivning} & \textbf{Datum} \\\hline
0.1 & Skapade projektplanen & 180925 \\\hline
\end{tabularx}
\end{table}

\section{Introduktion}
\subsection{Beskrivning}
Projektet går ut på att skapa en portföljhemsida där man kan
ladda upp och presentera färdigställda projekt tillsammans med bilder,
information, samt de tekniker som använts för att skapa projektet. Hemsidan
kommer att skapas med hjälp av micro-frameworket Flask som backend.

Hemsidan delas upp i två delar, ett datalager och ett presentationslager.

Projektet skapas som huvuddelen av kursen TDP003.

\subsection{Datalagret}
All information om de projekt som skall presenteras på hemsidan kommer
att ligga i en JSON-fil på servern. Datalagret kommer att interagera
med denna fil för att bland annat läsa in saker såsom tekniker som använts
i projektet i fråga, skapa listor av alla projekt genom en sökning och
dylikt.

Tekniker som används:
\begin{itemize}
\item Python3
\item Git
\item JSON
\end{itemize}

\subsection{Presentationslagret}
Efter att datalagret har läst in den information som begärts av användaren
i minne kommer presentationslagret att ta vidare för att skapa en
användarvänlig presentation på själva hemsidan. Hemsidan kommer bestå av
följande sidor enligt systemspecifikation:

\begin{itemize}
\item Förstasida med översikt och bilder från de senaste projekten
\item Söksida med lista över alla projekt samt med möjlighet till filtrering
\item Projektsida med fullständig information om det projekt som visas
\item Tekniksida med information projekt utefter använda tekniker
\end{itemize}
\newpage
Tekniker som används:
\begin{itemize}
\item Flask
\item Jinja2
\item Python3
\item Git
\item Bootstrap
\item JSON
\end{itemize}

\section{Arbetsplan}

\subsection{Arbetsmetodik}
Vi programmerar främst vid var sin dator och delar upp arbetet mellan
varandra på funktioner eller filer för att maximera effektiviteten.

Arbetet kommer främst att ske på plats i datasalarna och våra arbetstider
kommer med största sannolikhet att ligga mellan 10-17.

I början av varje dag möts vi och delar upp arbetet som ska ske den dagen
mellan varandra.
I slutet av varje arbetsvecka kommer vi stämma av så att vi ligger i fas med
vår planering. I veckor där vi har mycket att göra kommer vi eventuellt
stämma av med planeringen dagsvis istället för veckovis, för att förhindra
att vi hamnar för mycket ur fas gentemot våra uppsatta deadlines. Ifall vi
hamnar efter planeringen får en ny planering skapas där den tid som vi har
till övers får läggas på att komma ikapp. På det sättet förhindras att
en nedåtgående spiral skapas där vi hamnar ytterligare mer ur fas varje vecka.

Versionshanteringen kommer att ske via ett Git-repo.
\newpage
\section{Tidsplanering}

\subsection{Deadlines}
\begin{table}[!h]
\begin{tabularx}{\linewidth}{|l|X|}\hline
    \textbf{Datum} & \textbf{Moment} \\\hline
    18-09-13 & Planeringsdokument inlämnad. \\\hline
    18-09-20 & LoFi-prototyp inlämnad. \\\hline
    18-09-20 & Installationsmanual inlämmnad. \\\hline
    18-09-27 & Första utkast av projektplanen inlämnad. \\\hline
    18-09-27 & Första versionen av den gemensamma installationsmanualen klar. \\\hline
    18-10-04 & Bidrag till den gemensamma installationsmanualen eller testerna av portfolion inlämnade. \\\hline
    18-10-04 & Eventuella brister i den gemensamma installationsmanualen korrigerade. \\\hline
    18-10-04 & Eventuella brister i projektplanen korrigerade och senaste versionen inlämnad. \\\hline
    18-10-05 & Datalagret godkänt av assistent. \\\hline
    18-10-18 & Portfolion tillgänglig via OpenShift. \\\hline
    18-10-18 & Första versionen av systemdokumentationen inlämnad. \\\hline
    18-10-23 & Eventuella brister i systemdokumentationen korrigerade och senaste versionen inlämnad. \\\hline
    18-10-23 & Testdokumentation inlämnad. \\\hline
    18-10-23 & Individuellt reflektionsdokument inlämnat. \\\hline
\end{tabularx}
\end{table}
\subsection{Milstolpar}
\begin{table}[!h]
\begin{tabularx}{\linewidth}{|l|X|}\hline
    \textbf{Datum} & \textbf{Moment} \\\hline
    18-09-18 & 3/4 sidor av LoFi-prototypen klara och länkar till varandra. \\\hline
    18-09-18 & Utkastet till installationsmanualen ska kunna följas för att installera Flask och Jinja2. \\\hline
    18-09-25 & Mallen av projektpanen ska vara tillräckligt genomarbetad att text kan enkelt läggas till i respektive sektion. \\\hline
    18-10-02 & Alla funktioner i datalagret ska fungera utom search. \\\hline
    18-10-10 & Presentationslagret ska generera projektsidor och en tekniksida med en lista över alla projektsidor. \\\hline
\end{tabularx}
\end{table}


\subsection{Vecka 38}
\begin{table}[!h]
\begin{tabularx}{\linewidth}{|l|X|l|l|}
\hline
\textbf{Prio} & \textbf{Moment} & \textbf{Tidsåtgång} & \textbf{Färdigställt}\\\hline
1 & Färdigställa Lo-Fi prototyp & 5h & 18-09-19  \\\hline
2 & Göra grundläggande installationsmanual samt bekanta
oss med flask och jinja & 6-8h & 18-09-19 \\\hline
\end{tabularx}
\end{table}

\begin{table}[!h]
\begin{tabularx}{\linewidth}{|X|r|}\hline
    \textbf{Delmoment} & \textbf{ Planerad arbetsdag} \\\hline
    Skapa en HTML-sida för startsidan av LoFi-prototypen. & 18-09-17 \\\hline
    Skapa en HTML-sida för listsidan av LoFi-prototypen. & 18-09-17 \\\hline
    Skapa en HTML-sida för söksidan av LoFi-prototypen. & 18-09-18 \\\hline
    Skapa och färdigställa installationsmanualen. & 18-09-18 \& 18-09-19 \\\hline
    Skapa en HTML-sida för projektsidan av LoFi-prototypen. & 18-09-19 \\\hline
\end{tabularx}
\end{table}

\subsection{Vecka 39}
\begin{table}[!h]
\begin{tabularx}{\linewidth}{|l|X|l|l|}
\hline
\textbf{Prio} & \textbf{Moment} & \textbf{Tidsåtgång} & \textbf{Färdigställt}\\\hline
1& Utkastet av projektplanen färdigställt & 4h	& 18-09-27 \\\hline
2& Bidrag till gemensamma installationsmanualen & 2h &  18-09-28 \\\hline
\end{tabularx}
\end{table}

\begin{table}[!h]
\begin{tabularx}{\linewidth}{|X|r|}\hline
    \textbf{Delmoment} & \textbf{ Planerad arbetsdag} \\\hline
    Göra en mall för projektplanen. & 18-09-25 \\\hline
    Fylla mallen för att slutföra första utkastet. & 18-09-26 \& 18-09-27 \\\hline
    Göra ett bidrag till den gemensamma installationsmanualen. & 18-09-26 \\\hline
    Påbörja datalagret. Skapa en JSON-fil och bekanta oss med tekniken & 18-09-27 \& 18-09-28 \\\hline
\end{tabularx}
\end{table}
\subsection{Vecka 40}
\begin{table}[!h]
\begin{tabularx}{\linewidth}{|l|X|l|l|}\hline
\textbf{Prio} & \textbf{Moment} & \textbf{Tidsåtgång} & \textbf{Färdigställt}\\\hline
1& Datalagret testat och klart. & 2h & 18-10-04 \\\hline
2& Projektplan färdigställd och korrigerad. & 2h & 18-10-04 \\\hline
3& Searchfunktionen färdigställd & 4h & 18-10-03 \\\hline
4& Funktioner load, get\_project\_count, get\_project, get\_techniqes, get\_technique\_stats färdigställda. & 4h & 18-10-02 \\\hline
\end{tabularx}
\end{table}

\begin{table}[!h]
\begin{tabularx}{\linewidth}{|X|r|}\hline
    \textbf{Delmoment} & \textbf{ Planerad arbetsdag} \\\hline
    Ha en färdig datastruktur i JSON-filen så att vi kan skapa funktioner till den. & 18-10-01 \\\hline
    Skapa och färdigställa funktionerna load, get\_project\_count, get\_project, get\_techniques, get\_technique\_stats.  & 18-10-01 \& 18-10-02 \\\hline
    Skapa och färdigställa search-funktionen & 18-10-03 \\\hline
    Korrigera projektplanen och lämna in nästa version. & 18-10-03 \& 18-10-04 \\\hline
    Testa datalagret & 18-10-04 \\\hline
\end{tabularx}
\end{table}

\subsection{Vecka 41}
\begin{table}[!h]
\begin{tabularx}{\linewidth}{|l|X|l|l|}\hline
\textbf{Prio} & \textbf{Moment} & \textbf{Tidsåtgång} & \textbf{Färdigställt}\\\hline
1& Enklare sida med full funktionalitet byggd. & 30h & 18-10-12 \\\hline
2& Research av Bootstrap. & 3h & 18-10-08 \\\hline
\end{tabularx}
\end{table}

\begin{table}[!h]
\begin{tabularx}{\linewidth}{|X|r|}\hline
    \textbf{Delmoment} & \textbf{ Planerad arbetsdag} \\\hline
    Lära oss hur bootstrap fungerar. & 18-10-08 \\\hline
    Skapa och färdigställa de individuella projektsidorna. & 18-10-08 \& 18-10-09 \\\hline
    Skapa och färdigställa sidan med listorna över de olika projekten. & 18-10-09 \& 18-10-10 \\\hline
    Skapa och färdigställa söksidan. & 18-10-10 \& 18-10-11 \\\hline
    Skapa och färdigställa startsidan. & 18-10-11 \& 18-10-12 \\\hline
\end{tabularx}
\end{table}
\newpage
\subsection{Vecka 42}
\begin{table}[!h]
\begin{tabularx}{\linewidth}{|l|X|l|l|}
\hline
\textbf{Prio} & \textbf{Moment} & \textbf{Tidsåtgång} & \textbf{Färdigställt}\\\hline
1& Presentationslagret klart. Testat och fungerar med datalagret. & 10h & 18-10-17 \\\hline
2& Systemdokumentation färdigställd. & 8h & 18-10-17 \\\hline
3& Testdokumentation och reflektionsdokument färdigställt. & 15h & 18-10-22 \\\hline
4& Korrigera eventuella brister och fel i systemdokumentationen. & 3h & 18-10-22 \\\hline
\end{tabularx}
\end{table}

\begin{table}[!h]
\begin{tabularx}{\linewidth}{|X|r|}\hline
    \textbf{Delmoment} & \textbf{ Planerad arbetsdag} \\\hline
    Ändra utseendet på hemsidan så att den liknar LoFi-prototypen. & 18-10-15 \\\hline
    Skapa  en mall för systemdokumentation och skriva ett utkast för den. & 18-10-15 \\\hline
    Testa och färdigställa presentationslagret. & 18-10-16 \& 18-10-17 \\\hline
    Fyll systemdokumentationen och färdigställ den. & 18-10-16 \& 18-10-17 \\\hline
    Skriva testdokumentationen. & 18-10-18 \& 18-10-19 \& 18-10-22 \\\hline
    Skriva reflektionsdokumentet. & 18-10-19 \& 18-10-22 \\\hline
\end{tabularx}
\end{table}

\end{document}
