\documentclass{TDP003mall}
\definecolor{terminalgreen}{HTML}{8AE234}

\newcommand{\version}{Version 1.0}
\author{Eric Jönsson, \url{erijo137@student.liu.se}\\
  Ida Bergquist, \url{idabe112@student.liu.se}}
\title{Projektplan}
\date{2018-09-25}
\rhead{Ida Bergquist\\
Eric Jönsson}



\begin{document}
\projectpage
\tableofcontents
\newpage
\section{Revisionshistorik}
\begin{table}[!h]
\begin{tabularx}{\linewidth}{|l|X|l|}
\hline
\textbf{Ver.} & \textbf{Revisionsbeskrivning} & \textbf{Datum} \\\hline
0.1 & Skapade projektplanen & 180925 \\\hline
\end{tabularx}
\end{table}

\section{Introduktion}
\subsection{Beskrivning}
Projektet går ut på att skapa en portföljhemsida där man kan
ladda upp och presentera färdigställda projekt tillsammans med bilder,
information, samt de tekniker som använts för att skapa projektet. Hemsidan
kommer att skapas med hjälp av micro-frameworket Flask som backend.

Hemsidan delas upp i två delar, ett datalager och ett presentationslager.

Projektet skapas som huvuddelen av kursen TDP003.
\subsection{Datalagret}
All information om de projekt som skall presenteras på hemsidan kommer
att ligga i en JSON-fil på servern. Datalagret kommer att interagera
med denna fil för att bland annat läsa in saker såsom tekniker som använts
i projektet i fråga, skapa listor av alla projekt genom en sökning och
dylikt.

Tekniker som används:
\begin{itemize}
\item Python3
\item Git
\item JSON
\end{itemize}
\subsection{Presentationslagret}
Efter att datalagret har läst in den information som begärts av användaren
i minne kommer presentationslagret att ta vidare för att skapa en
användarvänlig presentation på själva hemsidan. Hemsidan kommer bestå av
följande sidor enligt systemspecifikation:
\begin{itemize}
\item Förstasida med översikt och bilder från de senaste projekten
\item Söksida med lista över alla projekt samt med möjlighet till filtrering
\item Projektsida med fullständig information om det projekt som visas
\item Tekniksida med information projekt utefter använda tekniker
\end{itemize}
Tekniker som används:
\begin{itemize}
\item Flask
\item Jinja2
\item Python3
\item Git
\item Bootstrap
\item JSON
\end{itemize}
\section{Arbetsplan}
\subsection{Arbetsmetodik}
Vi programmerar främst vid var sin dator och delar upp arbetet mellan
varandra på funktioner eller filer för att maximera effektiviteten.
\section{Tidsplanering}

\section{Vecka 38}
\begin{table}[!h]
\begin{tabularx}{\linewidth}{|l|X|l|l|}
\hline
\textbf{Deadline} & \textbf{Delmoment} & \textbf{Tidsåtgång} & \textbf{Färdigställt}\\\hline
18-09-20 & Färdigställa Lo-Fi prototyp & 2h & 18-09-17  \\\hline
18-09-20 & Göra grundläggande installationsmanual samt bekanta
oss med flask och jinja & 6-8h & 18-09-18 \\\hline
\end{tabularx}
\end{table}

\section{Vecka 39}
\begin{table}[!h]
\begin{tabularx}{\linewidth}{|l|X|l|l|}
\hline
\textbf{Deadline} & \textbf{Delmoment} & \textbf{Tidsåtgång} & \textbf{Färdigställt}\\\hline
18-09-27 & Utkast av projektplan & 10h & 18-09-26  \\\hline
18-09-27 & Första versionen av gemensamma installationsmanualen
& 3h & 18-09-26 \\\hline
18-10-05 & Påbörjan av datalagret & v39 & - \\\hline
\end{tabularx}
\end{table}
\newpage
\section{Vecka 40}
\begin{table}[!h]
\begin{tabularx}{\linewidth}{|l|X|l|l|}\hline
\textbf{Deadline} & \textbf{Delmoment} & \textbf{Tidsåtgång} & \textbf{Färdigställt}\\\hline
18-10-04 & Bidragit med förbättring till installationsmanualen
eller de gemensamman git-testerna.& 3h & 18-10-01  \\\hline
18-10-04 & Fixat eventuella brister i projektplan samt installationsmanualen & 1-10h & 18-10-03 \\\hline
18-10-05 & Datalagret genomarbetat och godkänt & v40 & 18-10-03 \\\hline
\end{tabularx}
\end{table}

\section{Vecka 41}
\begin{table}[!h]
\begin{tabularx}{\linewidth}{|l|X|l|l|}\hline
\textbf{Deadline} & \textbf{Delmoment} & \textbf{Tidsåtgång} & \textbf{Färdigställt}\\\hline
- & Arbete på presentationslagret & v41 & 18-10-17\\\hline
- & Arbete på systemdokumentation & v41 & 18-10-18\\\hline
\end{tabularx}
\end{table}

\section{Vecka 42}
\begin{table}[!h]
\begin{tabularx}{\linewidth}{|l|X|l|l|}
\hline
\textbf{Deadline} & \textbf{Delmoment} & \textbf{Tidsåtgång} & \textbf{Färdigställt}\\\hline
18-10-18 & Arbete med OpenShift och systemdokumentation & v42 & 18-10-17\\\hline
18-10-23 & Arbete med testdokumentation och individuellt reflektionsdokument & 15h & 18-10-22\\\hline
\end{tabularx}
\end{table}

\section{Vecka 43}
\begin{table}[!h]
\begin{tabularx}{\linewidth}{|l|X|l|l|}
\hline
\textbf{Deadline} & \textbf{Delmoment} & \textbf{Tidsåtgång} & \textbf{Färdigställt}\\\hline
18-10-23 & Korrigering av eventuella brister i systemdokumentationen & 1-10h & 18-10-22\\\hline
\end{tabularx}
\end{table}
\end{document}
a
