\documentclass{TDP003mall}
\definecolor{terminalgreen}{HTML}{8AE234}
\usepackage{graphicx} 
\usepackage{wrapfig}

\newcommand{\version}{Version 1.1}
\author{Eric Jönsson, \url{erijo137@student.liu.se}\\
  Ida Bergquist, \url{idabe112@student.liu.se}}
\title{Installationsmanual}
\date{2018-09-25}
\rhead{Ida Bergquist\\
Eric Jönsson}



\begin{document}
\projectpage
\section{Revisionshistorik}
\begin{table}[!h]
\begin{tabularx}{\linewidth}{|l|X|l|}
\hline
\textbf{Ver.} &\textbf{ Revisionsbeskrivning} &\textbf{ Datum}\\\hline
1.0 & Skapade installationsmanualen & 180919 \\\hline
1.1 & Lade till Felsökning och Projektinstallation, ändrade terminalkommandon till monospace & 180925 \\\hline
\end{tabularx}
\end{table}

\section{Förinstallationskrav}
Portfolion förlitar sig på följande program för att fungera korrekt:\\
\begin{enumerate}
\item Python3
\item Pip
\item Flask
\item Jinja2
\end{enumerate}
Se till att ha de senaste versionerna av programmen innandu installerar portfolion.\\
Nedan hittar du instruktioner för att installera och uppdatera programmen. Instruktionerna är skrivna för debian-baserade linuxdistrubitioner, ex. Ubuntu, Linux Mint etc\ldots
\subsection{Python}
Installera Python3 genom att skriva förjande kommando i terminalen:\\
\texttt{\textbf{\textcolor{terminalgreen}{user@computer}:\~{}\$ sudo apt install python3}}\\
Uppdatera python3 genom att skriva:\\
\texttt{\textbf{\textcolor{terminalgreen}{user@computer}:\~{}\$ sudo apt update \&\& sudo apt upgrade python3}}\\
Installera även python3-venv.\\
\texttt{\textbf{\textcolor{terminalgreen}{user@computer}:\~{}\$ sudo apt install python3-venv}}\\

\subsection{Pip}
Installera pip genom terminalen:\\
\texttt{\textbf{\textcolor{terminalgreen}{user@computer}:\~{}\$ sudo apt install python-pip}}\\

\newpage
\subsection{Flask och Jinja2}
\begin{wrapfigure}{r}{0.4\textwidth}
    \begin{center}
        \includegraphics[width=0.40\textwidth]{./list2.png}
    \end{center}
\end{wrapfigure}
Innan Flask installeras behöver du skapa en mapp där du vill installera portfolion. Gå sedan in i den mappen och starta terminalen. Högerklicka i mappen och tryck på 'Open in terminal' i menyn. Sedan behöver du ska en virtuell miljö för att portfolion inte ska gå sönder vid framtida uppdateringar. Skriv följande kommando i terminalen:\\
\texttt{\textbf{\textcolor{terminalgreen}{user@computer}:\~{}\$ python3 -m venv venv}}\\
Flask och Jinja2 installeras via pip. Skriv följande kommando i terminalen:\\
\texttt{( venv ) \textbf{\textcolor{terminalgreen}{user@computer}:\~{}\$ pip install Flask}}\\
När Flask installeras ingår flera paket, bl.a. Jinja2.
För att kontrollera att du nu har både Flask och Jinja2 kan du skriva följande i terminalen:\\
\texttt{( venv ) \textbf{\textcolor{terminalgreen}{user@computer}:\~{}\$ pip list}}\\
Kontrollera att både Flask och Jinja2 finns med i listan.
\section{Installation för Linux}
Öppna mappen du vill installera portfolion i. Högerklicka i mappen och tryck på "Open in terminal". Skriv sedan följande kommando i terminalen:\\
\texttt{( venv ) \textbf{\textcolor{terminalgreen}{user@computer}:\~{}\$ git clone git@gitlab.ida.liu.se:erijo137/Portfolioprogram}}\\
Du ska nu kunna installera portfolion genom att skriva följande i terminalen:\\
\texttt{( venv ) \textbf{\textcolor{terminalgreen}{user@computer}:\~{}\$ pip install -e .}}\\
\section{Felsökning}
Om du får felmeddelande under något steg under installationen, läs felmeddelandet noga. Ifall du saknar något paket så kommer felmeddeandet oftast berätta vad som saknas och vilket kommando du ska skriva för att införskaffa det. Generellt sätt kan de flesta fel lösas med att göra installationen från början. \\
Om din fil inte finns i portfolion, kontrollera att du följt projektinstallationen med rätt format på ditt projekt.
\section{Projektinstallation}
För att installera ett projekt ska du lägga till information om ditt projekt i en JSON-fil. Det är inte bestämt än hur vi vill ha det formaterat men det kommer i den gemensamma installationsmanualen.
\end{document}
